\chapter{Conclusioni}
\label{cha:conclusioni}

 In questo lavoro di tesi è stata illustrata la metodologia utilizzata per ricavare informazioni dal caricamento di immagini tramite i social network. Nello specifico, abbiamo mostrato come diverse tipologie di device e interfacce software della stesa piattaforma (WhatsApp) lascino tracce distintive sulle immagini condivise che possono essere identificate ed esaminate per costituire prove dal punto di vista forense. Siamo partiti nel capitolo introduttivo presentando il campo di ricerca della \textit{social media forensics}, i quesiti a cui cerca di dare una risposta, i rami in cui si divide con una breve descrizione e gli obiettivi da raggiungere. Successivamente è stata fatta una rassegna dei principali lavori che costituiscono lo stato dell'arte della \textit{platform provenance analysis}. In particolare, sono stati suddivisi in tre categorie: quelli che hanno contribuito alla creazione di nuovi dataset, quelli che si sono occupati di estrarre \textit{feature} dai media condivisi e infine quelli che hanno utilizzato algoritmi di \textit{machine learning} per classificare i dati a disposizione, con una riflessione sui possibili sviluppi futuri di questo ambito. Per indagare lo scenario definito siamo partiti dalla costruzione di SHADE, dataset costituito da immagini condivise dalla stessa piattaforma ma tramite modalità differenti. A questo punto abbiamo iniziato ad eseguire le analisi sui dati raccolti, prima effettuando un confronto qualitativo e in seguito, grazie ai \textit{feature descriptors}, abbiamo estratto diverse tipologie di informazioni dalle immagini. Da ultimo, è stata effettuata la classificazione impiegando algoritmi di \textit{machine learning} adeguatamente allenati. I risultati ottenuti mostrano come la maggior parte delle classi analizzate sia correttamente identificabile, in alcuni casi raggiungendo livelli di accuratezza molto elevati.\\
 Questo lavoro è stato utilizzato per realizzare un articolo (\cite{tomasoni2022device}) che sarà presentato alla conferenza \textit{IEEE} del 2022 inerente alla \textit{Multimedia Signal Processing}. L'argomento trattato nella tesi potrà essere esteso in futuro impiegando molteplici piattaforme di condivisione e considerando problemi in cui vengono analizzati contestualmente immagini e video, in modo da poter definire e risolvere scenari che rappresentano la vita reale.\newpage